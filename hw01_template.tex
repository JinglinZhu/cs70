\title{CS 70 Spring 2018 Homework 1}
\documentclass{article}\usepackage{amsmath,amssymb,amsthm,tikz,tkz-graph,color,chngpage,soul,hyperref,csquotes,graphicx,floatrow, yfonts}\newcommand*{\QEDB}{\hfill\ensuremath{\square}}\newtheorem*{prop}{Proposition}\renewcommand{\theenumi}{\alph{enumi}}\usepackage[shortlabels]{enumitem}\usepackage[nobreak=true, framemethod=tikz]{mdframed}\usetikzlibrary{matrix,calc, automata, positioning}\MakeOuterQuote{"}\usepackage[margin=1in]{geometry} \newtheorem{theorem}{Theorem}
\usepackage{tabto}
    \NumTabs{20}
\usepackage{fancyhdr}
%New commands courteous of Anonymous
\newcommand{\Real}{\mathbb{R}}
\newcommand{\real}{\Real}
\newcommand{\Realsp}{\mathbb{R}^+}
\newcommand{\realsp}{\Realsp}
\newcommand{\Cmplx}{\mathbb{C}}
\newcommand{\complex}{\Cmplx}
\newcommand{\cmplx}{\Cmplx}
\newcommand{\Rats}{\mathbb{Q}}
\newcommand{\rats}{\Rats}
\newcommand{\Nats}{\mathbb{N}}
\newcommand{\nats}{\Nats}
\newcommand{\Ints}{\mathbb{Z}}
\newcommand{\ints}{\Ints}
\newcommand{\Intsp}{\mathbb{Z}^+}
\newcommand{\intsp}{\Intsp}
%%%
\newcommand{\xor}{\mathbin{\oplus}}
\pagestyle{fancy}
\rfoot{your name here | your SID here} % change these

\headheight=40pt

\renewcommand{\headrulewidth}{6pt}
\lhead{ \Large  \fontfamily{lmdh}\selectfont
CS 70   \tab Discrete Mathematics and Probability Theory
\\Spring 2018       \tab    \tab Babak Ayazifar, Satish Rao}
\rhead{\LARGE   \fontfamily{lmdh}\selectfont    Homework 1} %change HW number here


\begin{document}

% Below is a template for variously formatted problems and solutions. 
% I've included examples of formatting as well (a kind of tutorial). 

\section*{Sundry}
Before you start your homework, write down your team. Who else did you work with on this
homework? List names and email addresses. (In case of homework party, you can also just describe
the group.) How did you work on this homework? Working in groups of 3-5 will earn credit for
your "Sundry" grade.
\begin{flushleft}
Please copy the following statement and sign next to it:\linebreak\linebreak
\textit{I certify that all solutions are entirely in my words and that I have not looked at another student’s
solutions. I have credited all external sources in this write up.}
\end{flushleft}

\begin{mdframed}
    \textit{Solution}
\end{mdframed}

\section*{1. Propositional Practice}
Convert the following English sentences into propositional logic and the following propositions into English.  State whether or not each statement is true with brief justification.
\begin{enumerate}

  \item There is a real number which is not rational.
  \begin{mdframed}
  $\exists n  \in R, n \notin Q$ \par
  True. \par
  Proof: We proceed by cases. Let $n = \sqrt{2}$, then n is a real number, but not a rational one. Thus the statement holds.
  \end{mdframed}

  \item All integers are natural numbers or are negative, but not both.
  \begin{mdframed}
   $\forall n \in \mathbb{Z}, ((n \in \mathbb{N}) \lor (n \in \mathbb{Z}^-)) \land \neg ((n \in \mathbb{N}) \lor (n \in \mathbb{Z}^-))$ \par
   True. \par
   Proof: We proceed by contradiction. Assume $\exists x \in \mathbb{Z}$, $\neg ((n \in \mathbb{N}) \lor (n \in \mathbb{Z}^-)) \lor ((n \in \mathbb{N}) \land (n \in \mathbb{Z}^-))$  \par
   Since $\mathbb{N} \cup \mathbb{Z}^- = \mathbb{Z}$, the statement $\neg ((n \in \mathbb{N}) \lor (n \in \mathbb{Z}^-))$, which equals $n \notin (\mathbb{N} \cup \mathbb{Z}^-)$ is false.\par
   Since $\mathbb{N} \cap \mathbb{Z}^- = \varnothing$, the statement $((n \in \mathbb{N}) \land (n \in \mathbb{Z}^-))$, which equals $n \in (\mathbb{N} \cap \mathbb{Z}^-)$is false.\par
   So the contradiction is false. Thus the statement is true.
  \end{mdframed}

  \item If a natural number is divisible by 6, it is divisible by 2 or it is divisible by 3.
  \begin{mdframed}
  $ n \in \mathbb{N} $, $(6 \: | \: n) \implies (2 \: | \: n) \lor (3 \: | \: n)$ \par
  True. \par
  Proof: We proceed by direct proof. Since $(6 \: | \: n)$, n can be written as n = 6q for $q \in \mathbb{Z}$. Thus n = 2(3q) for $3q \in \mathbb{Z}$, meaning $(2 \: | \: n)$. n = 3(2q) for $2q \in \mathbb{Z}$, meaning $(3 \: | \: n)$.
  \end{mdframed}

  \item $(\forall x \in \mathbb{R})\ (x \in \mathbb{C})$
  \begin{mdframed}
  All real numbers are complex numbers. \par
  True. \par
  Proof: We proceed by direct proof. Since $\mathbb{R} \subseteq \mathbb{C}$, then $\forall x \in \mathbb{R}, x \in \mathbb{C}$
  \end{mdframed}

  \item $(\forall x \in \mathbb{Z})\ ((2 \mid x \lor 3 \mid x) \implies 6 \mid x)$
  \begin{mdframed}
  If an integer is divisible by 2 or by 3, it is divisible by 6. \par
  False. \par
  Proof: We proceed by cases. 4 is divisible by 2, but not divisible by 6. Thus the statement is false.
  \end{mdframed}

  \item $(\forall x \in \mathbb{N})\ ((x > 7) \implies ((\exists a, b \in \mathbb{N})\ (a + b = x)))$
  \begin{mdframed}
   Any natural number larger than 7 can be written as the sum of two natural numbers. \par
   True. \par
   Proof: We proceed by induction. \par
   Base case(x = 8): 8 = 2 + 6. The statement holds. \par
   Inductive Hypothesis: For arbitrary $k \geq 8$, $\exists a, b \in \mathbb{N}$, a + b = k. \par
   Inductive Step: k + 1 = a + b + 1 = a + (b + 1). $a, (b + 1) \in \mathbb{N}$. Thus the statement holds. \par
   By the principle of induction, the statement is true.
  \end{mdframed}

\end{enumerate}

\pagebreak

\section*{2. Miscellaneous Logic}
\begin{enumerate}
\item Let the statement, $\forall x \in \mathbb{R}, \exists y \in \mathbb{R} \  G(x,y)$, be
true for predicate $G(x,y)$. 

For each of the following statements, decide if the statement is certainly true, certainly false, or possibly true, and justify your solution.

\begin{enumerate}[(i)]

  \item
  $G(3,4)$
  \begin{mdframed}
   Possibly true. \par
   Assume there is only $y = y^*$ that satisfies $ \forall x \in \mathbb{R} \ G(x,y^*)$. Then if $y^* \neq 4$, the statement is false. If $y^* = 4$, the statement is true.
  \end{mdframed}

  \item
  $\forall x \in \mathbb{R}$ $G(x,3)$
  \begin{mdframed}
  Possibly true. \par
   Assume there is only $y = y^*$ that satisfies $ \forall x \in \mathbb{R} \ G(x,y^*)$. Then if $y^* \neq 3$, the statement is false. If $y^* = 3$, the statement is true.
  \end{mdframed}

  \item
  $\exists y$ $G(3,y)$
  \begin{mdframed}
  Certainly true.
  Set x = 3. Then according to the original statement, $\exists y$ $G(3,y)$ is true.
  \end{mdframed}

  \item
  $\forall y$ $\neg G(3,y)$
  \begin{mdframed}
  Certainly false.
  The contradiction of this statement is statement(iii), which is true. So this statement is false.
  \end{mdframed}

  \item
  $\exists x$ $G(x,4)$
  \begin{mdframed}
  Possibly True.
  Assume there is only $y = y^*$ that satisfies $ \forall x \in \mathbb{R} \ G(x,y^*)$. Then if $y^* = 4$, G(x, y) holds for any x. If $y^* \neq 4$, it's not sure if there exists an x that satisfies G(x, y).
  \end{mdframed}

\end{enumerate}

\item Give an expression using terms involving $\lor,\land$ and $\neg$ which is true if and only if
exactly one of $X,Y$, and $Z$ are true.  (Just to remind you: $(X \land Y \land Z)$ means
all three of $X$,$Y$,$Z$ are true, $(X \lor Y \lor Z)$ means at least one of $X$,$Y$
and $Z$ is true.)
\begin{mdframed}
$ (X \land (\neg Y) \land (\neg Z)) \lor ((\neg X) \land Y \land (\neg Z)) \lor ((\neg X) \land (\neg Y) \land Z) $
\end{mdframed}

\end{enumerate}

\pagebreak

\section*{3. Prove or Disprove}
\begin{enumerate}

  \item $\forall n \in \mathbb{N}$, if $n$ is odd then $n^2 + 2n$ is odd.
  \begin{mdframed}
   We proceed by direct proof. \par
   n = 2k + 1 for $ k \in \mathbb{Z} \implies n^2 + 2n = (4k^2 + 4k + 1) + 2(2k + 1) = 4k^2 + 8k +3 = 2(k^2 + 4k + 1) + 1$. Since $(k^2 + 4k + 1)$ is an integer, then $2(k^2 + 4k + 1)$ is even, thus $2(k^2 + 4k + 1) + 1$ is odd. The statement holds.
  \end{mdframed}

  \item $\forall x, y \in \mathbb{R}$, $\min(x, y) = (x + y - |x - y|)/2$.
  \begin{mdframed}
  We proceed by cases. \par
  If $x < y, min(x, y) = (x + y - (y - x)) / 2 = x$. The statement holds. \par
  If $y < x, min(x, y) = (x + y - (x - y)) / 2 = y$. The statement holds. \par
  If $x = y, min(x, y) = (x + y ) / 2 = x \ or \ y$. The statement holds. \par
  Thus the statement is true.
  \end{mdframed}

  \item $\forall a, b \in \mathbb{R}$ if $a + b \le 10$ then $a \le 7$ or $b \le 3$.
  \begin{mdframed}
  We proceed by contraposition. \par
  Assume $(a > 7) \cap (b > 3)$, then $a + b > 10$. The contraposition is true.
  Thus the statement is true.
  \end{mdframed}

  \item $\forall r \in \mathbb{R}$, if $r$ is irrational then $r + 1$ is irrational.
  \begin{mdframed}
  We proceed by contraposition. \par
  Assume $r + 1$ is rational, then $r + 1$ can be written as $ r + 1 = a / b $ for $ a, b \in \mathbb{N} , b \neq 0$ (a and b share no common factor other than 1).
  $r = a / b - 1 = (a - b) / b$. \par
  Prove $(a - b)$ and $b$ share no common factor other than 1: We proceed by contraposition. Assume $(a - b)$ and $b$ have common factor $q$, then $(a - b) = nq, b = mq$ for $n, m \in \mathbb{Z}$. Then $a = (m + n)q$ shares common factor q with b. Thus the statement $(a - b)$ and $b$ share no common factor other than 1 holds. 
  Since $(a - b)$ and $b$ share no common factor other than 1, r is rational. The contraposition got proved. \par
  Thus the original statement holds.
  \end{mdframed}

  \item $\forall n \in \mathbb{N}$, $10n^2 > n!$.
  \begin{mdframed}
  We proceed by cases.
  If $n = 0$, $10 \times 0^2  = 0 < 0! = 1$. Thus the statement is false.
  \end{mdframed}

\end{enumerate}

\pagebreak

\section*{4. Preserving Set Operations}
Define the image of a set $X$ to be the set $f(X) = \{y~|~y = f(x) \text{ for some } x \in X\}$. Define the inverse image of a set $Y$ to be the set $f^{-1}(Y) = \{x~|~f(x) \in Y\}$. Prove the following statements, in which $A$ and $B$ are sets. By doing so, you will show that inverse images preserve set operations, but images typically do not.

\textit{Hint: For sets $X$ and $Y$, $X=Y$ if and only if $X \subseteq Y \text{ and } Y \subseteq X$. To prove that $X \subseteq Y$, it is sufficient to show that $\forall x,~x \in X \implies x \in Y$.}

\begin{enumerate}
    \item $f^{-1}(A \cup B) = f^{-1}(A) \cup f^{-1}(B)$.
    \begin{mdframed}
 Prove $ f^{-1}(A \cup B) \subseteq f^{-1}(A) \cup f^{-1}(B): \forall x, f(x) \in (A \cup B) \implies (f(x) \in A) \lor (f(x) \in B) \implies x \in \{x~|~f(x) \in A\} \cup \{x~|~f(x) \in B\}$ \par
 Prove $ f^{-1}(A) \cup f^{-1}(B) \subseteq f^{-1}(A \cup B): \forall x, (f(x) \in A) \lor (f(x) \in B) \implies f(x) \in A \cup B \implies x \in \{x~|~f(x) \in A \cup B\}$ \par
 Thus, the statement $f^{-1}(A \cup B) = f^{-1}(A) \cup f^{-1}(B)$ got proved.
    \end{mdframed}

    \item $f^{-1}(A \cap B) = f^{-1}(A) \cap f^{-1}(B)$.
    \begin{mdframed}
  Prove $ f^{-1}(A \cap B) \subseteq f^{-1}(A) \cap f^{-1}(B): \forall x, f(x) \in (A \cap B) \implies (f(x) \in A) \land (f(x) \in B) \implies x \in \{x~|~f(x) \in A\} \cap \{x~|~f(x) \in B\}$ \par
  Prove $ f^{-1}(A) \cap f^{-1}(B) \subseteq f^{-1}(A \cap B): \forall x, (f(x) \in A) \land (f(x) \in B) \implies f(x) \in A \cap B \implies x \in \{x~|~f(x) \in A \cap B\}$ \par
  Thus, the statement $f^{-1}(A \cap B) = f^{-1}(A) \cap f^{-1}(B)$ got proved.
    \end{mdframed}
    
    \item $f^{-1}(A \setminus B) = f^{-1}(A) \setminus f^{-1}(B)$.
    \begin{mdframed}
    Prove $f^{-1}(A \setminus B) \subseteq f^{-1}(A) \setminus f^{-1}(B)$: $ \forall x, f(x) \in \{ f(x) \in B \ | \ f(x) \notin A \} \implies (f(x) \in A) \land (f(x) \notin B) \implies x \in \{ x~|~f(x) \in A \} \land x \in \{ x~|~f(x) \notin B \}  \implies x \in f^{-1}(A) \setminus f^{-1}(B)$ \par
    Prove $f^{-1}(A) \setminus f^{-1}(B) \subseteq f^{-1}(A \setminus B)$: $ \forall x,  x \in \{ x~|~f(x) \in A \} \land x \in \{ x~|~f(x) \notin B \} \implies (f(x) \in A) \land (f(x) \notin B) \implies f(x) \in \{ f(x) \in B \ | \ f(x) \notin A \} \implies x \in f^{-1}(A \setminus B)$ 
    \end{mdframed}
    
    \item $f(A \cup B) = f(A) \cup f(B)$.
    \begin{mdframed}
     Prove $f(A \cup B) \subseteq f(A) \cup f(B)$: $ \forall y $ that $ y = f(x), x \in A \cup B \implies (y = f(x), x \in A) \lor (y = f(x), x \in B) \implies y \in f(A) \cup f(B)$ \par
     Prove $f(A) \cup f(B) \subseteq f(A \cup B)$: $ \forall y $ that $ (y = f(x), x \in A) \lor (y = f(x), x \in B) \implies y = f(x), x \in A \cup B \implies  y \in f(A \cup B)$
    \end{mdframed}
    
    \item $f(A \cap B) \subseteq f(A) \cap f(B)$, and give an example where equality does not hold.
    \begin{mdframed}
     Proof: $ \forall y $ that $ (y = f(x)) \land (x \in A \cup B) \implies ((y = f(x)) \land ((x \in A) \land (x \in B))) \implies ((y = f(x)) \land (x \in A)) \land ((y = f(x)) \land (x \in B)) \implies y \in f(A) \cap f(B)$ \par
     Example: Let $A = \{-1\}$, $B = \{1\}$, $f(x) = x^2$. Thus $f(A \cap B) = \varnothing, f(A) \cap f(B) = \{ 1 \}$
     
    \end{mdframed}
    
    \item $f(A \setminus B) \supseteq f(A) \setminus f(B)$, and give an example where equality does not hold.
    \begin{mdframed}
     Proof: $ \forall y $ that $ ((y = f(x)) \land (x \in A)) \land ((y = f(x)) \land (x \in B)) \implies (y = f(x)) \land ((x \in A) \land (x \notin B)) \implies y \in f(A \setminus B)$
    \end{mdframed}
    
\end{enumerate}

\pagebreak

\section*{5. Hit or Miss?}
State which of the proofs below is correct or incorrect.
For the incorrect ones, please explain clearly where the logical error in the proof lies.
Simply saying that the claim or the induction hypothesis is false is \emph{not} a valid explanation of what is wrong with the proof.
You do not need to elaborate if you think the proof is correct.

\begin{enumerate}
    \item
        \textbf{Claim:} For all positive numbers $n \in \mathbb{R}$, $n^2 \ge n$.
        \begin{proof}
            The proof will be by induction on $n$.  \\
            \emph{Base Case:} $1^2 \ge 1$. It is true for $n=1$.    \\
            \emph{Inductive Hypothesis:} Assume that $n^2 \ge n$.   \\
            \emph{Inductive Step:} We must prove that $(n+1)^2 \ge n+1$.
            Starting from the left hand side,
            \begin{align*}
                (n+1)^2 &= n^2+2n+1 \\
                        &\ge n+1.
            \end{align*}
            Therefore, the statement is true.
        \end{proof}

    \begin{mdframed}
    Incorrect. \par
    This proof only proves the case for all positive integers . It misses those $n \in \mathbb{R} \setminus \mathbb{Z^+}$
    \end{mdframed}

    \item
        \textbf{Claim:} For all negative integers $n$, $-1-3-\cdots+(2n+1) = -n^2$.
        \begin{proof}
            The proof will be by induction on $n$.  \\
            \emph{Base Case:} $-1 = -(-1)^2$. It is true for $n=-1$. \\
            \emph{Inductive Hypothesis:} Assume that $-1-3-\cdots+(2n+1) = -n^2$. \\
            \emph{Inductive Step:} We need to prove that the statement is also true for $n-1$ if it is true for $n$, that is,
             $-1-3-\cdots+(2(n-1)+1) = -(n-1)^2$. Starting from the left hand side,
            \begin{align*}
                -1-3-\cdots+(2(n-1)+1)
                &= (-1-3-\cdots + (2n+1))+(2(n-1)+1)    \\
                &= -n^2 + (2(n-1)+1)                \quad \text{(Inductive Hypothesis)}\\
                &= -n^2 + 2n - 1                    \\
                &= -(n-1)^2.
            \end{align*}
            Therefore, the statement is true.
        \end{proof}

    \begin{mdframed}
    Correct.
    \end{mdframed}

    \item
        \textbf{Claim:} For all nonnegative integers $n$, $2n=0$.
        \begin{proof}
            We will prove by strong induction on $n$.   \\
            \emph{Base Case:} $2 \times 0 = 0$. It is true for $n=0$.   \\
            \emph{Inductive Hypothesis:} Assume that $2k=0$ for all $0 \le k \le n$.    \\
            \emph{Inductive Step:} We must show that $2(n+1)=0$. Write $n+1 = a+b$ where $0 < a,b \le n$.
            From the inductive hypothesis, we know $2a = 0$ and $2b=0$, therefore,
            \begin{align*}
                2(n+1) = 2(a+b) = 2a + 2b = 0+0 =0.
            \end{align*}
            The statement is true.
        \end{proof}
        
    \begin{mdframed}
    Incorrect. \par
    The inductive step, "$n + 1 = a + b $ where $  0 < a, b \leq n$" is wrong. If $n = 0$, then $n + 1 = a + b$, where one of a and b must equal 0. Let $a = 0, b = 1$, then $b = n$. Since now we only have $2 \times 0 = 0$ proved, we cannot prove $2b = 0$, which is $2\times1 = 0$.
    \end{mdframed}
\end{enumerate}

\pagebreak

\section*{6. Badminton Ranking}
A team of $n$ ($n \geq 2$) badminton players held a tournament, where every person plays with every other person exactly once, and there are no ties. Prove by induction that after the tournament, we can arrange the $n$ players in a sequence, so that every player in the sequence has won against the person immediately to the right of him.
\begin{mdframed}
 We proceed by induction on n. \par
 Let $W(A, B)$ denote the fact that A wins B. \par
 Base Case($n = 2$): Assume $W(A, B)$, then put $B$ on the right of $A$. The statement holds. \par
 Inductive Hypothesis: Assume the statement holds true for $n$ players. $(n \geq 2)$  \par
 Inductive Step: There are now $n + 1$ players. Let the new player be $X$. \par 
 According to the inductive hypothesis, we have $n$ players arranged in the required sequence. \par 
 If we can find players $A, B$ standing next to each other that satisfy $W(A, X), W(X, B)$, we can just put X between A and B.\par
 If we cannot find such $A, B$ that satisfy $W(A, X), W(X, B)$, then here are two cases, either X wins everyone or everyone wins X. If X wins everyone, then just place X on the left of the whole sequence. If everyone wins X, then place X on the right of the whole sequence.\par
 Thus the statement got proved.
\end{mdframed}

%Based off of EE16A design
%Compiled and edited by Stephan Kaminsky (11:08 AM 1/20/2018)
\end{document}